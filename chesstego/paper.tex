\documentclass[twocolumn]{article}
\usepackage[top=1.1in, left=0.85in, right=0.85in]{geometry}

\usepackage{url}
% \usepackage{code}
% \usepackage{cite}
\usepackage{amsmath}
\usepackage{amssymb}
\usepackage{graphicx}
\usepackage{chessboard}

\pagestyle{empty}

\usepackage{ulem}
% go back to italics for emphasis, though
\normalem

\usepackage{natbib}

% \newcommand\comment[1]{}
\newcommand\sfrac[2]{\!{}\,^{#1}\!/{}\!_{#2}}

\begin{document} 

\title{It still seems that black has hope in these extremely unfair variants of chess}
\author{Dr.~Tom~Murphy~VII~Ph.D. \and Ben Blum \and
Jim McCann, in italics ...... oh! .... nervous laughter .... no, that's not part of the name still. I stopped saying the name. Why are you still typing? This isn't the name any more. End quote.\thanks{
Copyright \copyright\ 2014 the Regents of the Wikiplia
Foundation. Appears in SIGBOVIK 2014 with the undivided
attention of the Association for Computational Heresy; 
{\em IEEEEEE!} press, Verlag-Verlag volume no.~0x40-2A.
\$0.00}}

\setchessboard{showmover=false}

\newcommand\checkmate{\hspace{-.05em}\raisebox{.4ex}{\tiny\bf ++}}

\renewcommand\th{\ensuremath{{}^{\textrm{th}}}}
\newcommand\st{\ensuremath{{}^{\textrm{st}}}}
\newcommand\rd{\ensuremath{{}^{\textrm{rd}}}}
\newcommand\nd{\ensuremath{{}^{\textrm{nd}}}}
\newcommand\at{\ensuremath{\scriptstyle @}}

\date{1 April 2014}

\maketitle \thispagestyle{empty}

\begin{abstract}
CHECKMATE.
\end{abstract}

\section*{Introduction}

Chess is an old-timey game that you already know. One problem with
Chess is that it is hard; both players may struggle mightily in a
game, expending their brain-sugars, and it is not clear who the winner
will be. Another problem with Chess is that it isn't other games, and
we're pretty much over it. In this paper we attempt to address both
problems, with limited success. We show how to combine Chess with
several other board games, in order to make it more predictable.

As usual for a Tom~7 SIGBOVIK joint, the results herein are real.
The source code that was used to solve the games or prove that no
winning strategy exists up to some depth can be found on the inter-net.\!\footnote{In the Subversion repository at:
  \url{https://sourceforge.net/p/tom7misc/svn/HEAD/tree/trunk/chesstego/}}

\section{Chesstego}

{\em Chesstego} is a combination of the games Chess and Stratego. You
already should know that every game in this paper is a combination of
Chess and something, but I wanted to emphasize the portmanteau. All of
the games will be named with portmanteau, and some of the names will
be achingly bad.

In Stratego, each player begins the game by arranging his or her
Stratego-pieces in a secret fashion on the board. In the world of
Stratego, civization is ruled by a leader known as Flag. The player's
goal is to assassinate the opponent Flag, using a member of his or her
army. However, which piece represents Flag is unknown!

There are many ways we might apply the Stratego system of governance
to Chess.

\paragraph{Chesstego variation I.} In this variation, each player decides
in secret, before the game begins, which of his or her pieces is the
actual King, that is, the Flag. If this piece is checkmated, the game
is lost. There are two subvariations: {\em I(a)}, where a player must
announce {\em Check!} when the Flag is under attack, and all of the
normal rules about moving into (or castling through) check must be
obeyed. In subvariation {\em I(b)}, which I prefer, the piece that is
Flag may slip silently into and out of check, and the game only ends
when that piece is captured.\footnote{In Chess proper, these
  formulations are nearly equivalent---except for rules like castling
  through check and some corner cases---but it is deemed important for
  movie drama that players be able to announce {\em Check!} and {\em
    Checkmate!} at times.}

{\em Chesstego variation I} is a reasonable if slightly silly game,
and it is difficult. It may even be more difficult than Chess due to the
psychological mind-games that are possible. It can be hard to tell which
player is winning, let alone which player will win. In this paper we
are interested in variants of Chess that both {\em are other games} and
{\em are predictable}. We'd like to give one of the players a clear
strategy for winning.

\paragraph{Chesstego variation II.} In this variation, each player
decides in secret, before the game, which of her {\em opponent's} pieces
is Flag. Same as before, if this piece is captured, the game ends
instantly. But now, players don't even know which of their own pieces
is their glorious ruler, Flag. This can be very exciting or tittilating.

Unfortunately, there is no known winning strategy for White, and there
are no winning strategies with fewer than 6 moves. This was proved
by computer program. In fact, it was proved for a stronger case:

\paragraph{Chesstego variation III.} In this variation, the first player,
known as White, chooses in secret both the identity of Flag for his own
populace, as well as the identity of Flag for his opponent. Even in
this very unfair setup, Black can always survive for at least 6 moves.

% TODO: Example of black escaping.

This will not do! Black just has so many options with all those pieces.
Perhaps if he were handicapped somewhat?

\paragraph{Chesstego variation IV.} In this variation, White chooses
the identity of both Flags again, in secret without telling her
opponent, and also Black does not get any good pieces, just pawns. Like this:

\chessboard[
  setfen=4k3/pppppppp/8/8/8/8/PPPPPPPP/RNBQKBNR]

Lo! This version is finally satisfactory. White shall pick one of her
safe pieces (e.g., one of the rooks) as Flag, and choose Black's b7~pawn
as Black~Flag. White's winning move is c2c4 and then Qb3, with Black~Flag
unable to escape the B file. Choosing the f7~pawn works as well.

% make chesstego.exe && ./chesstego.exe -blackpieces justking 1 > king1.tex
% \input{king1}

Example hopeless match: 1. c4 Kf8~~2. Qb3 b5~~3. c4xb4\checkmate \\
\chessboard[setfen=5k2/p1pppppp/8/1P6/8/1Q6/PP1PPPPP/RNB1KBNR]

\paragraph{Chesstego variation V.} This variation is just like {\em IV}
but White gets super good pieces everywhere instead of having some
dumb ones, and Black still gets bupkiss:

\chessboard[
  setfen=4k3/pppppppp/8/8/8/8/NNNBBNNN/QQQQKQQQ]

Now there are many more winning strategies, but the one from {\em IV}
works as well.

\paragraph{Other variations.} Incidentally, my on-line version of Chess 
called SICO\footnote{ On the internet at:
  \url{http://snoot.org/toys/sico/}}---where you just play a single
move in a random game---has for about six years had variations called
``center wall'' and ``barricades''\cite{sico} which were inspired by
the layout of the Stratego board. However, these cannot be considered
worthy of portmanteau, as the rules are basically just Chess rules.

% Chessy Crush Saga

\section{Cluess}

This game, known as {\em Chuessdo} in the United Kingdom, is a cross
between Chess and {\em Clue}. In this version, the player called White
is known as Mrs.~White, and the player called Black is known as
Professor~Plum.



\setchessboard{tinyboard,showmover=false}
% \chessboard[
%   setfen=rnbqkbnr/pppppppp/8/8/8/8/NNNBBNNN/QQQQKQQQ]

% \input{king2}

% results:

% No mate in 12 for 0-7. 6 ran for like 2 full days!
% regular board.

% Regular board still:
% No mate in 14 for 1,2,3,4,5,7. didn't run 6 because 6-ply took
% 2 full days!
% 0 finally finished with no mates found.

% No mate in 10 for king at pawn index 0-7, even when
% white's pieces are all queens (but the king)
% (6 also finished last. Is this the weakest spot?)

% No mate in 10 for white having ``mega'' pieces

% Mate in 10 for pawn idx 0! 441450 tree 
%                 and idx 1! 210316 tree

% TODO: We always have to explore all of black's moves, but we
% can push the search much deeper (if we just want to prove the
% existence of mate) by prioritizing white's moves. e.g., on the
% last move, only move to capture the king; and on the second-to-last,
% only move into check.

% TODO: Generate ``book'' where white is making strong moves
% (black has few options) but black keeps it going for 10/12 moves?

\bibliography{paper}{}
\bibliographystyle{plain}

\end{document}

% rnbqkbnr/pppppppp/8/8/8/8/NNNBBNNN/QQQQKQQQ w KQkq - 0 1
