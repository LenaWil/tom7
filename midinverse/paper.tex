\documentclass[twocolumn]{article}
\usepackage[top=1.1in, left=0.85in, right=0.85in]{geometry}

\usepackage{url}
% \usepackage{code}
% \usepackage{cite}
\usepackage{amsmath}
\usepackage{amssymb}
\usepackage{graphicx}

\pagestyle{empty}

\usepackage{ulem}
% go back to italics for emphasis, though
\normalem

\usepackage{natbib}

\newcommand\comment[1]{}
\newcommand\sfrac[2]{\!{}\,^{#1}\!/{}\!_{#2}}

\begin{document} 

\title{What is the opposite of a song?}
\author{Dr.~Tom~Murphy~VII~Ph.D.\thanks{
Copyright \copyright\ 2014 the Regents of the Wikiplia
Foundation. Appears in SIGBOVIK 2014 with whatever the
opposite of permission of the 
Association for Computational Heresy is; {\em IEEEEEE!} press,
Verlag-Verlag volume no.~0x40-2A.
\$0.00}
}

\renewcommand\th{\ensuremath{{}^{\textrm{th}}}}
\newcommand\st{\ensuremath{{}^{\textrm{st}}}}
\newcommand\rd{\ensuremath{{}^{\textrm{rd}}}}
\newcommand\nd{\ensuremath{{}^{\textrm{nd}}}}
\newcommand\at{\ensuremath{\scriptstyle @}}

\date{1 April 2014}

\maketitle \thispagestyle{empty}

\begin{abstract}
what do I put here
\end{abstract}

\vspace{1em}
{\noindent \small {\bf Keywords}:
  musical instrument digital inverse,
  ...
}

\section*{Introduction}

Opposite Day, which is known in mathematics as {\it duality}, is a
powerful technique for turning good things into bad things. For
example, it can turn the perfectly straightforward reasoning technique
of {\it induction} into the real head-scratcher {\it coinduction}, or
the righteous and valid {\it true} into the muschevious and degenerate
{\it false}. Dualization can also turn bad things into good, of
course. In this paper, I investigate what kinds of formal notions of
duality can be employed to answer the question: What is the opposite
of a song?

\paragraph{Note.} Not since the 80s when in extreme novelty circumstances a magazine or
Happy Meal would come with a crap phonograph engraved into flimsy acetate has
it been possible to include audio material in printed conference
proceedings, and as you might imagine, the work presented here does
involve sound. Therefore I recommend reading this paper with
simultaneous access to the audio recordings at {XXX TODO}, or watching
the video summary linked to from that site. For those who've become
marooned on a desert isle with the complete SIGBOVIK proceedings, the
hearing-impaired, post-human societies attempting to recreate 2014-era
science and technology from the academic record of prestigious
conferences, or the moderately lazy, I have included over-the-top
textual descriptions of the songs in this work.

% ideas: Graphical transpose of piano roll -- long notes (horizontal
% bar) become many short notes played at once (vertical bar).

% convert MIDI to bitmap and perform invertible photoshop-like filters.
% the above becomes a 90\deg rotation of 127-tick-long chunks, for
% example.

% AES encryption?!

\section{What is a song?}

It should be easy to find dozens of wankerholes who have tried to
define ``song'' and some other art jokers who have pushed the
boundaries of these definitions. The author approves of this
mischief---the prions of music---but this paper is not concerned with
that tired analysis. For the purposes of this paper, we'll use only a
simple and practical high-level encoding of music, MIDI.

\paragraph{MIDI.} It stands for Musical Instrument Digital Interface
and you can tell from the need to clarify that it is ``digital'' that
it is as old as old dead dry dust and just barely adequate for modern
applications, and so has never been successfully superseded. It
permits 16 {\it channels}, each more-or-less a separate
instrument,\!\footnote{It's really more like 16 members of an
  orchestra, each of whom can only play one instrument at a time, but
  we'll make the additional simplification in this paper that each
  channel corresponds to a single instrument.} with 128 {\it notes}
per channel. Each of the 128 MIDI notes has a fixed meaning, from 0 =
C$_{-1}$, which is 5 octaves below middle C (about 8~Hz\footnote{A
  note like ``middle C'' has a fixed semantics in some wishy-washy
  music sense, but the actual frequency of the tone is determined by
  mapping that meaning into one of several conventional scales. For
  example, in ``concert pitch'', middle A is fixed to 220 Hz (so
  middle C is 261.63~Hz), but there is also ``computer science
  pitch'', where middle C is 256~Hz. The techniques here are mostly
  insensitive to the absolute pitch values, but we use computer
  science pitch because what the hey. }) to 127 = G9 (about
12,274~Hz). A track consists of a time-ordered series of events, with
the main events being ``turn note $n$ on with at volume $v$'' and
``turn note $n$ off'';\footnote{I still haven't decided how I feel
  about this, but the note-off event also includes a velocity, which I
  guess means something like ``how hard you yank your finger off the
  piano key''. It is usually ignored and in fact the convention
  actually seems to encode note-off as ``note-on with volume 0''. In
  this paper I just use note-off to mean whatever concrete event
  results in the note being inaudible.} volumes also range 0--127,
allowing for every possible nuance of expression. Notes can be turned
on and off arbitrarily, but we restrict ourselves to the sensible
subset where a note-on event always precedes exactly one note-off
event for the same note on the same track, with no nesting or overlaps
or other shenanigans. For such MIDIs, there exists a simple
isomorphism between the event-based representation and a ``bitmap''
representation where for each MIDI tick, each note in each track has a
volume level 0--127, where 0 is predominant and means ``off''. Of
course there are a jillion other oddities to MIDI, but this will
suffice as a definition of a ``song'' for now.

%  127 g11
%  126 f#
%  125 f
%  124 e 
%  123 d# 
%  122 d 
%  121 c#
%  120 c11
%  108 c10
%  96  c9 
%  84  c8
%  72  c7
%  60 = middle c
%  48  c5
%  36  c4
%  24  c3
%  12  c2
%  0   c1

\section{What is an opposite?}

There are many formal and informal ideas of ``opposite''. Here I will
use a very loose interpretation of opposite so that we can explore
many different reasonable ideas. We define not the opposite of a
specific thing, but a method for getting the opposite of all of the
things of some type. So an opposite of some thing of type $A$ will be a
total function $f$ of type $A \rightarrow A$ where
%
\begin{itemize}
\item $\exists x{:}A. f(x) \neq x$
\item $\forall x{:}A. f(f(x)) = x$
\end{itemize}

The first condition simply says that the opposite function must not be
the identity function; there must be some things that are not their
own opposites. Requiring that nothing be its own opposite is quite a
strong condition and excludes some things we would normally think of
as opposites. (For example, it seems reasonable to say the opposite of
$3$ is $-3$, but for negation, the opposite of $0$ is still $0$.) For
songs that are their own opposites, see
Section~\ref{sec:selfopposites}.

The second condition is that f is its own inverse, which expresses
the idea that items ``pair up'' into their opposites. So if the opposite
of day is night, then we expect the opposite of night to be day. This
is called an ``involution'' in mathematics.

In the case of this paper, the type $A$ will be ``song'' (MIDI file)
and so we'll look at a bunch of different functions $f$ for turning
one MIDI file into another. If we run the function on a MIDI file and
get its opposite, then run it again to get the opposite of that thing,
we'll be back where we started, like if there were an Opposite Hour
within Opposite Day where everything is normal.

\section{Well-known opposites}

Others have asked this question before (XXX cite beatles, etc.). One
idea is so old we'll just mention it and move on: Playing the song
in reverse. (XXX cover it briefly?)

% any rotation mod 127 (except 0)

\section{Self-opposites} \label{sec:selfopposites}

\section{Conclusion}


\bibliography{paper}{}
\bibliographystyle{plain}
\end{document}
