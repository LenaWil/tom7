\documentclass[twocolumn]{article}
\usepackage{fullpage}
\usepackage{amssymb}
\usepackage{fancyvrb}
\usepackage{url}
\usepackage{proof}

\begin{document}

\newcommand\wcite[1]{\footnote{\tiny Wikipedia, the free encyclopedia: {\it #1}; 2007}}
\newcommand\comment[1]{}
\newcommand\z{\ensuremath{\!}}

\newcommand\D\Delta
\newcommand\G\Gamma
\newcommand\m\mapsto
\newcommand\eval{{\sf eval\,\,}}
\newcommand\tag[2]{{\tt{\mbox{\tt <}}{#1}{\mbox{\tt>}}}{#2}{\tt{\mbox{\tt</}}{#1}{\mbox{\tt>}}}}
\newcommand\prim{{\,\,\sf prim}}
\newcommand\rate{{\sf rate\,\,}}

\title{{\bf \huge Wikiplia}:\\
       The Free Programming Language\\
       That Anyone Can Edit}

\author{Tom Murphy VII}
\date{1 April 2007}

\maketitle

\begin{abstract}
We present a new programming language called Wikiplia. The language
has an unprecedented level of integration: The system is its own
compiler, language definition, documentation, development environment,
distributed filesystem, database, revision control system, software
license, community message board, and World Wide Web home site.
Wikiplia is designed to be Free to a greater extent and in more
dimensions than existing languages.
\end{abstract}

\section{Introduction}

One of the most cherished social principles of mankind is
freedom,\z\wcite{Freedom} in its many incarnations. More recently,
freedom has become an important principle in computer science as well,
with the introduction of Free Software licenses such as the GNU
GPL,\z\wcite{GNU General Public License} extensible markup languages
such as XML,\z\wcite{XML} the ability expicitly deallocate memory
with the {\tt free(3)} library call, and the widespread availibility of
Free Herbal V1agara on the World Wide Web.\z\wcite{Sildenafil}

The aim of this project is to develop a programming language that is
as free as possible. We begin by enumerating freedoms that we desire
to support. Because freedom is a possession of inestimable
value\wcite{Cicero} we do not attempt to rank these freedoms; instead,
each freedom is ``numbered'' using a symbol drawn from incomparable
sets of glyphs.

\paragraph{Freedom $\alpha$: The freedom to tinker.}
Users should be able to study a program to see how it works, and to
make modifications to suit his or her needs. For most software, this
means that the programmer needs access to the software's
documentation, source code, and UML\wcite{Unified Modeling Language}
use case diagrams. This is traditionally achieved through licenses
such as the GPL; however, as we will discuss in
Section~\ref{sec:bootstrap} there are special considerations for
bootstrapping compilers that render the GPL inadequate for this
purpose.

% freedom of speech?

\paragraph{Freedom $\Game$: Freedom of beer.}
Users should be able to write software without paying money to a
licensing authority or certification program.

\paragraph{Freedom $\surd$: Freedom of USA \#1.}
Wikiplia is 100\% made in the USA and only available in
English.\z\wcite{Freedom fries}

% XXX more...

\section{Reflections on booting boots} \label{sec:bootstrap}

\section{Core calculus} \label{sec:calculus}

\begin{figure}[ht]

\comment{
    List of exp list       <list> ... </list>
  | String of string       <string>s</string>
  | Int of IntInf.int      <int>i</int>
  | Symbol of string       <symbol>s</symbol>
  (* the only lazy expression *)
  | Quote of exp           <quote>exp</quote>
  (* user can't write these *)
  | Prim of prim           <prim>p</prim>
  | Closure of (string * exp) list * string * exp
}
\begin{center}
\begin{tabular}{rcl}
 {\em X} & ::= & {\tt<}{\em tag}{\tt >} $X_1\ X_2\ \ldots\ X_n$ {\tt</}{\em tag}{\tt >} \\
         & $|$ & string
\end{tabular}
\end{center}
\caption{Syntax of XML} \label{fig:xml}
\end{figure}

\begin{figure*}[ht]
\[\begin{array}{cc}
% basic, easy
\infer{\G \vdash \eval \tag{string}{s} \m \tag{string}{s}}{} &
\infer{\G \vdash \eval \tag{quote}{X} \m X}{} \\[1em]
\infer{\G \vdash \eval \tag{int}{s} \m \tag{int}{s}}{} &
\infer{\G \vdash \eval \tag{prim}{s} \m \tag{prim}{s}}{} \\[1em]
\infer{\G \vdash \eval \tag{symbol}{s} \m \tag{prim}{s}}{s \prim} &
\infer{\G \vdash \eval \tag{symbol}{s} \m X}{\G(s) = X} \\[1em]
\multicolumn{2}{c}{\infer{\G \vdash \eval \tag{closure}{\G\ s\ X} \m \tag{closure}{\G\ s\ X}}{}} \\[1em]
\multicolumn{2}{c}{\infer{\G \vdash \eval \tag{list}{X_1\ \ldots\ X_n} \m X'}%
{\G\vdash \eval X_1 \m X_1' & \cdots & \G\vdash \eval X_n \m X_n' &
 \G\vdash \rate X_1'\ \ldots\ X_n' \m X'
}} \\[1em]
% todo:
% skip? abort, handle

%
\end{array}\]

\caption{Evaluation of XML, part 1. The judgment $\G \vdash \eval X \m X'$
indicates an assessment of the document $X$ with value $X'$. The
judgment $\rate$ is an auxiliary assessment of a list of documents. It is defined in Figure~\ref{fig:xmlrate}.
$\vec{X}$ is shorthand for a possibly empty sequence of XML documents.
$\G$ is itself an XML document of the form
% XXX is this right wrt <symbol>?
$\tag{list}{\tag{symbol}{s_1}\ X_1\ \ldots\ \tag{symbol}{s_n}\ X_n}$.
We take the judgment $\G(s) = X$ to produce the leftmost $X_i$ in $\G$
such that $s_i$ is $s$. $\G, s = X$ is $\tag{list}{\tag{symbol}{s}\ X
\vec{X}}$ if $\G$ is $\tag{list}{\vec{X}}$. 
%
} \label{fig:xmleval}
\end{figure*}

\begin{figure*}[htp]
\[\begin{array}{cc}
% evaluation of lists
\multicolumn{2}{c}{\infer{\G \vdash \rate \tag{prim}{\sf list}\ \vec{X} \m \tag{list}{\vec{X}}}{}} \\[1em]
\multicolumn{2}{c}{\infer{\G \vdash \rate \tag{prim}{\sf cons}\ X\ \tag{list}{\vec{X}} \m \tag{list}{X\ \vec{X}}}{}} \\[1em]
\multicolumn{2}{c}{\infer{\G \vdash \rate \tag{prim}{\sf lambda}\ \tag{symbol}{s}\ X \m \tag{closure}{\G\ s\ X}}{}} \\[1em]
\infer{\G \vdash \rate \tag{closure}{\G'\ s\ X}\ \vec{X} \m X'}{\G',s = \tag{list}{\vec{X}} \vdash \eval X \m X'} &
\infer{\G \vdash \rate \tag{prim}{\sf eval}\ X \m X'}{\G \vdash \eval X \m X'} \\[1em]
\multicolumn{2}{c}{\infer{\G \vdash \rate \tag{prim}{\sf xcase}\ \tag{list}{}\ X\ \vec{X} \m X'}{\G \vdash \eval X \m X'}} \\[1em]
\multicolumn{2}{c}{\infer{
\begin{array}{r@{}l}
  \G \vdash \rate & \tag{prim}{\sf xcase}\ \tag{list}{X_h \vec{X_t}}\  X_0\ \\
                  & \tag{list}{\tag{symbol}{s_h}\ \tag{symbol}{s_t}\ X_b}\ \vec{X} \m X' \\
\end{array}
}{\G, s_h = X_h, s_t = X_t \vdash \eval X_b \m X'}} \\[1em]
\multicolumn{2}{c}{\infer{
\begin{array}{r@{}l}
  \G \vdash \rate & \tag{prim}{\sf xcase}\ \tag{quote}{X}\  X_0\ X_1\ \\
                  & \tag{list}{\tag{symbol}{s_q}\ X_b}\ \vec{X} \m X' \\
\end{array}
}{\G, s_q = X \vdash \eval X_b \m X'}} \\[1em]
\multicolumn{2}{c}{\infer{
\begin{array}{r@{}l}
  \G \vdash \rate & \tag{prim}{\sf xcase}\ \tag{string}{s}\  X_0\ X_1\ X_2\ 
                    X\ \vec{X} \m X' \\
\end{array}
}{\G \vdash \eval X \m X'}} \\[1em]
\multicolumn{2}{c}{\infer{
\begin{array}{r@{}l}
  \G \vdash \rate & \tag{prim}{\sf xcase}\ \tag{int}{s}\  X_0\ X_1\ X_2\ X_3\ 
                    X\ \vec{X} \m X' \\
\end{array}
}{\G \vdash \eval X \m X'}} \\[1em]
\multicolumn{2}{c}{\infer{
\begin{array}{r@{}l}
  \G \vdash \rate & \tag{prim}{\sf xcase}\ \tag{symbol}{s}\  X_0\ X_1\ X_2\ X_3\ X_4\ \\
                  & \tag{list}{\tag{symbol}{s_s}\ X_b}\ \vec{X} \m X' \\
\end{array}
}{\G, s_s = \tag{string}{s} \vdash \eval X_b \m X'}} \\[1em]
\multicolumn{2}{c}{\infer{
\begin{array}{r@{}l}
  \G \vdash \rate & \tag{prim}{\sf xcase}\ \tag{t}{\vec{X_t}}\  X_0\ X_1\ X_2\ X_3\ X_4\ X_5\ X \vec{X} \m X' \\
\end{array}
}{t = {\tt prim} \,\,or\,\, {\tt closure} & \G \vdash \eval X \m X'}} \\[1em]
\multicolumn{2}{c}{\infer{\G \vdash \rate \tag{prim}{\sf quote}\ {X} \m \tag{list}{X}}{}} \\[1em]
%
\multicolumn{2}{c}{\infer{
\G \vdash \rate \tag{prim}{\sf head}\ \tag{string}{s} \m X
}{{\tt cvs\ checkout}\ s = X}} \\[1em]
\multicolumn{2}{c}{\infer{
\G \vdash \rate \tag{prim}{\sf read}\ \tag{string}{s} \tag{int}{i} \m X
}{{\tt cvs\ checkout\ -\!r}\ i\ s = X}} \\[1em]
\multicolumn{2}{c}{\infer{
\G \vdash \rate \tag{prim}{\sf insert}\ \tag{string}{s} X \m \tag{int}{i}
}{{\tt cvs\ commit}\ s\ X = i}} \\[1em]
\multicolumn{2}{c}{\infer{
\G \vdash \rate \tag{prim}{\sf history}\ \tag{string}{s} \m \tag{list}{\vec{X}}
}{{\tt cvs\ log}\ s = \vec{X}}} \\[1em]

\end{array}\]

\caption{The definition of the {\sf rate} judgment. The rules for
rating the primitives {\sf string}, {\sf sub}, {\sf substr}, {\sf +},
{\sf -}, {\sf int}, {\sf eq} and {\sf if} are omitted for space. 
%
} \label{fig:xmlrate}
\end{figure*}


Wikiplia is built upon a core calculus of structured data with
primitive revision control. Because we wish to support the freedom to
tinker, the structured data take the form of XML (the {\em extensible}
markup language; Figure~\ref{fig:xml}). Similar to the W3C's XML
Validation,\z\wcite{XML schema} we allow the quality of an XML
document to be assessed via a process called {\em evaluation}, whose
output (if any) is itself an XML document. The rules for XML
evaluation are simple and are given in
Figures~\ref{fig:xmleval}~and~\ref{fig:xmlrate}.



\comment{
  \nocite{alice}


  {\small
  \bibliographystyle{plain}
  \bibliography{wikiplia}
  }
}

\end{document}