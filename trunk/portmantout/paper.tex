\documentclass{article}
\usepackage[top=0.25in, left=0.25in, right=0.25in, bottom=0.25in]{geometry}

\usepackage{setspace}
\usepackage{seqsplit}
\usepackage{amsmath}
\usepackage{amssymb}
% \usepackage{code}
\usepackage{graphicx}
\usepackage{fancyvrb}
\usepackage{url}
\usepackage{textcomp}

\pagestyle{empty}

\newcommand\comment[1]{}

\newcommand\st{$^{\mathrm{st}}$}
\newcommand\nd{$^{\mathrm{nd}}$}
\newcommand\rd{$^{\mathrm{rd}}$}
\renewcommand\th{$^{\mathrm{th}}$}
\newcommand\tm{$^{\mbox{\tiny \textsc{tm}}}$}

\newcommand\sfrac[2]{{}\,$^{#1}$\!/{}\!$_{#2}$}

\newcommand\citef[1]{\addtocounter{footnote}{1}\footnotetext{\cite{#1}}\ensuremath{^{\mbox{\footnotesize [\thefootnote]}}}}
\begin{document} 

\title{The Portmantout}
\author{Dr.~Tom~Murphy~VII~Ph.D.\thanks{
Copyright \copyright\ 2015 the Regents of the Wikiplia
Foundation. Appears in SIGBOVIK 2015 with the {\bf etaoin shrdlu} of the
Association for Computational Heresy; {\em IEEEEEE!} press,
Verlag-Verlag volume no.~0x40-2A.
BTC~0.00
}
}

\renewcommand\>{$>$}
\newcommand\<{$<$}

\date{1 April 2015}

\maketitle

\newcommand\foo[1]{%
  \begin{minipage}{7.6in}
  \seqsplit{#1}
  \end{minipage}
  }

\section{Introduction}

A {\em portmanteau}, henceforewith non-italicized, is a
stringin'-together of two words to make a new word, like
``brogrammer'' ({\sf brother} + {\sf programmer}; a programmer who is
your brother), ``hupset'' ({\sf hungry} + {\sf upset}; a bit more
passive than hangry), or ``webinar'' ({\sf web} + {\sf nerd}).
Portmanteaus were invented by Lewis Carroll, the Jabberwock of
wordplay.

It is natural to think of generalizations of the portmanteau, such as
the {\em portmantrois},\!\footnote{Graham Smith, personal
  communication.} (itself a portmanteau of {\sf portmanteau} + {\sf
  trois}, French-language for three) the human-centipedification of
{\em three} words, such as ``anachillaxis'' ({\sf anaphylaxis} + {\sf
  chill} + {\sf relax}; a severe allergic reaction to idleness) or
``brogrammermaid'' ({\sf brother} + {\sf programmer} + {\sf mermaid};
a programmer who is your brother and a mermaid).

In this paper I present the world's first (?)\footnote{I did a couple
  Google searches; seems good enough.}\footnote{Enjoy source code: {\tt http://sourceforge.net/p/tom7misc/svn/HEAD/tree/trunk/portmantout}} {\em portmantout}, a
portmanteau of all English-language words ({\em tout} means ``all'' in
French-language). I also considered calling this a {\em portmantotal},
{\em portmantotale}, etc., as well as {\em portmantoutal} (a
portmantroix of the first three) or even {\em portmantoutale}. You
kind of see how this can get out of hand. The word is
%% XXX check final version
630,408
%% XXX
letters long and contains all 108,709 words in a particular wordlist
called {\tt wordlist.asc}.\footnote{Tom Murphy VII, ``What words ought
  to exist?'', SIGBOVIK~2011} Even though nobody can really agree
what all the words in English are, the technique used to generate this
portmantout should work for most very long word lists, although we will
see in Section~\ref{sec:join} that a handful of words are very important.

Since the word itself is
%% XXX check final version
11 pages
%% XXX
long in 4pt type with $.75$ linespacing, and this SIGBOVIK proceedings is positively overfull
hbox with content, we should probably get on with it.

\section{Computationalizing ``portmanteau''}

A real portmanteau is usually phonetic, like ``portmantotally'' is
about the sound of ``---teau'' being the same as ``to---''. It's also
not unusual for part of the word to be completely dropped, as in
``chillax'', which drops the ``re---'' from {\sf relax}. They are also
usually clever or meaningful. For the sake of computing a portmantout,
we need to make some rules about what it is, and it can't require
cleverness or semantic/phonemic interpretation of words if I'm going to
start and complete this project on the day of the SIGBOVIK deadline.

{\bf Generalized portmanteau.}\quad For a set of strings $L$, a string
$s$ is a generalized portmanteau if the entire string can be covered
by strings in $L$. A cover is a set of word occurrences $W = \langle m, n \rangle$,
where $s_m$--$s_n$ (the substring starting at offset $m$ and ending at $n$, inclusive)
is in $L$, and, taken sorted by the $m$ component, $W_i.n >= W_{i+1}.m$ for
each $i$ in range. For example,

\qquad\qquad\qquad\qquad\qquad\qquad\qquad\qquad\qquad\includegraphics[width=2in]{rewrotempered}


This string can be covered by {\sf rewrote}, {\sf temper}, and {\sf
  red}, so it is a generalized portmanteau if these three strings are
in $L$. (Spoiler alert: $L$ is English-language, so they are.)
Importantly, the covering strings overlap: {\sf rewro}, {\sf temper}
and {\sf ed} on their own would not cover this string (and {\sf rewro}
is not a word). Therefore, we cannot simply concatenate the entire
dictionary.

{\bf Portmantout.}\quad A generalized portmanteau is a portmantout if
it contains every string in $L$ as a substring. The words need not be
in its cover, as there may be muliple covers (In fact I conspicuously
did not choose the simpler one {\sf rewrote} + {\sf tempered}.) Other
words, like {\sf wrote}, {\sf rote} and {\sf rot} are in there
too ``for free'', even though they may not be able to participate in
a legal cover.

A word may appear multiple times; we just have to get them all. This
is fairly unavoidable---the word {\sf a} appears
%% XXX check final
60,374 times
% XXX
in the portmantout. Perhaps more surprising is that the word {\sf
  iraq} appears
%%% XXX check final
315 times.
%% XXX


Note that a ``generalized'' portmanteau does not actually include most
colloquial portmanteaus; {\sf brogrammer} cannot be covered since we
dropped the ``p---'' in {\sf programmer}. {\sf brogrammar} is also not
a generalized portmanteau since {\sf bro} and {\sf grammar} do not
overlap, yo, but I think it would be accepted colloquially by most
dudes. Disrupt!

\section{Generating a portmantout}

It's fairly straightforward handwaving to see that generating the
shortest portmantout is NP-complete. Seeing that it is in NP is easy;
we just need to check the cover and look up all the substrings, which
is clearly polynomial. It is probably NP-hard because the traveling
salesman problem can be reduced to it; for each node in the graph,
generate a two-symbol word $xy$ where $x$ and $y$ are fresh symbols;
for each edge between cities $x_iy_i$ and $x_jy_j$ generate a string
$y_iu^{w}x_j$, where $u$ is also a fresh symbol repeated $w$ times,
the cost of the edge. This allows us only to join two city words by
using a corresponding edge word.\footnote{There are some rubs: TSP
  requires that nodes only be visited once but a portmantout can use
  words multiple times. I believe that the multi-visit generalization
  of TSP is also NP-hard. The portmantout solution also requires visiting
  every edge, but we can relax this by concatenating all edge words $e$ into a
  new mega-long word like $e_0 z e_1 z \ldots e_k$ where $z$ is also
  a fresh symbol; since this word must appear and all edges are substrings
  of it, we now have no requirement that the rest of the solution (containing
  our TSP embedding) contains all edge words. This kind of trick also lets us
  set the start node for TSP.}

OK but good news! Since it's NP-complete, we know that we can come up
with a solution that's both non-optimal and slow, and we can still
feel pretty good about it. We proceed in two steps: Generating
particles eagerly, and then joining them together.

\subsection{Generating particles} \label{sec:gen}

For the first step, we load up all the words, and insert each word
into a multimap, keyed by each of its non-empty prefixes. We then
start by initializing a particle from any word; we choose {\sf
  portmanteau} to start, obviously. Then, repeatedly:
\begin{itemize}
\item Check each suffix of the particle in descending length-order,
\begin{itemize}
  \item If we have a word that has not already been used, strip the suffix from its start and append the remainder to the particle
\end{itemize}
\item Otherwise, emit the particle and start a new one with any unused word.
\end{itemize}

As an additional optimization, we discard words that are
substrings of any particle. This search makes the program much slower,
but it produces a much more efficient portmantout.

This always makes progress, using up one word at each step: We either
append it to our current particle, or we start a new particle with a
word. The particles are all generalized portmanteaus by construction,
because each added word has non-empty overlap with the previous one.
Here's an example particle: {\sf overmagnify\-ingearlesshrimp\-ierabbinicalci\-cadaeratorshrimpiestandard\-izablease\-holdershrimp\-inge\-mentshrimpsychedelically} ({\sf overmagnifying} + {\sf gearless} + {\sf shrimpier} + {\sf rabbinical} + {\sf calc} + {\sf cicada} + {\sf aerators} + {\sf shrimpiest} + {\sf standardizable} + {\sf leaseholders} + {\sf shrimping} + {\sf impingements} + {\sf shrimps} + {\sf psychedelically}; presumably having something to do with shrimp).

At this point, we have about 38,000 particles, many of which are a
single word. English contains many imbalances, like vastly more words
ending with ``---y'' (10,071) than beginning ``y---'' (only 338), so
it is not hard to see how we may get stuck with no new words to add to
a particle. We've also used each word only once, and locally maximized
the amount of overlap. If we can join these particles together in a
valid way, we'll have a portmantout.

\subsection{Joining particles} \label{sec:join}

Since we've already used every word, and, by construction, these
particles cannot be adjoined directly, we know we will need to reuse
some words to join them together. A simple way to do this is to
construct a table of size $26^2$ that for every pair of letters $a$
and $b$, contains a short word that starts with $a$ and ends with $b$.
86\% of the table entries can be filled in, but some letters are very
tricky: For example, almost no words end with ``q'' (we have only {\sf
  colloq} and {\sf iraq} in our dictionary), and there are no words
that start with ``v'' and end with ``f''. Fortunately, if we consider
all two-word (generalized) portmanteaus, using basically the same
algorithm as in Section~\ref{sec:gen}, we can fill the table
completely (Figure~\ref{fig:table}).

It is lucky for the existence of words like {\sf iraq}; they are used
for many of the entries in the ``q'' column. In fact, without a
handful of such words, it might be the case that English would not
permit a portmantotal! There are probably some less irregular
languages that cannot achieve this lexical feat. \verb+:'-(+

This table alone would allow us a very simple algorithm for generating
a valid portmantotal: Just take words from the dictionary and
concatenate them, but when putting e.g. {\sf tv} and {\sf farm}
together, we use the v--f linker {\sf vetof} ({\sf veto} + {\sf of})
from the table. We can't ever fail! However, this would produce a
portmantout that's bigger than the dictionary itself, which isn't very
economical. Rather than use the whole dictionary this way, we instead
join all of the 38,000 particles from the previous section. These are
much more compact. And now we are done!

\section{The portmantout}

This portmantout is
%% XXX check final version
630,408
%% XXX
letters long; there are 931,823 total letters in the dictionary so this
is a compresion ratio of 
%% XXX
1.47:1.
%% XXX
In comparison, ``brogrammermaid'' (although an illegal generalized
portmanteau) has a compression ratio of \sfrac{24}{14} = 1.71:1. So it
is fair to say that we are in the ballpark of a ``solid portmanteau.''
Of course, the gold standard is a compression ratio of $\infty : 1$---for
the case that we have the word {\sf portmanteau}, a totally overlapping portmanteau of
{\sf portmanteau} + {\sf portmanteau},\!\footnote{Cara Gillotti, personal communication, 2015.} iterated infinitely.

\begin{figure}[htp]
\input{table-two}
\caption{Minimal joining strings for every letter (rows) to every other letter (columns).} \label{fig:table}
\end{figure}

%  \begin{sloppypar}
%  \tiny
%  \setstretch{0.1}
%  \hyphenpenalty=1
%  \pretolerance=2
%  \tolerance=2
%  
%  \noindent \input{portmantout}
%  \end{sloppypar}

\end{document}
